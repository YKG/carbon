%-----------------------------------------------------------------------------
%
%               Template for sigplanconf LaTeX Class
%
% Name:         sigplanconf-template.tex
%
% Purpose:      A template for sigplanconf.cls, which is a LaTeX 2e class
%               file for SIGPLAN conference proceedings.
%
% Guide:        Refer to "Author's Guide to the ACM SIGPLAN Class,"
%               sigplanconf-guide.pdf
%
% Author:       Paul C. Anagnostopoulos
%               Windfall Software
%               978 371-2316
%               paul@windfall.com
%
% Created:      15 February 2005
%
%-----------------------------------------------------------------------------


\documentclass[preprint, cm]{sigplanconf}

% The following \documentclass options may be useful:

% preprint      Remove this option only once the paper is in final form.
% 10pt          To set in 10-point type instead of 9-point.
% 11pt          To set in 11-point type instead of 9-point.
% authoryear    To obtain author/year citation style instead of numeric.

\usepackage{amsmath}


\begin{document}

\special{papersize=8.5in,11in}
\setlength{\pdfpageheight}{\paperheight}
\setlength{\pdfpagewidth}{\paperwidth}

\conferenceinfo{CONF 'yy}{Month d--d, 20yy, City, ST, Country}
\copyrightyear{20yy}
\copyrightdata{978-1-nnnn-nnnn-n/yy/mm}
\doi{nnnnnnn.nnnnnnn}

% Uncomment one of the following two, if you are not going for the
% traditional copyright transfer agreement.

%\exclusivelicense                % ACM gets exclusive license to publish,
                                  % you retain copyright

%\permissiontopublish             % ACM gets nonexclusive license to publish
                                  % (paid open-access papers,
                                  % short abstracts)

%\titlebanner{banner above paper title}        % These are ignored unless
%\preprintfooter{This project is supported by a CCF-Tencent
%Open Research Fund, number CCF-Tencent AGR20130105.}   % 'preprint' option specified.

\title{Carbon: an Open Framework for Analysis and Transformation of Android Applications}
%\subtitle{Subtitle Text, if any}

\authorinfo{Name1 \and Name2 \and Name3}
           {School of Software Engineering\\
           University of Science and Technology of China}
           {Email1}
%\authorinfo{Name2\and Name3}
%           {Affiliation2/3}
%           {Email2/3}

\maketitle

\begin{abstract}
In recent years, Android platform has emerged as one of the most
important open mobile platforms. However, as Android has introduced
new file format and new Dalvik instruction set, existing binary-level
program analysis and transformation tools can not be used directly on
Android application binaries.

This paper presents the design and implementation of Carbon: a
new open framework for analysis and transformation of Android
applications at binary level. First, Carbon is an intermediate
representation suitable for program analysis and transformation, in
its current status, Carbon contain both high-level IR such as
abstract syntax trees, and low-level IR such as control-flow graphs. Second, Carbon
contains a front-end which can parse any given binary APK files and produce
abstract syntax trees and control-flow graphs. Third, Carbon is
now being used in several of
our ongoing projects, such as a symbolic executor for Dalvik; our
preliminary experiments and experience shows that Carbon is
an effective Android application binary analysis and transformation
framework, and we believe it can usefully be applied in other
situations as well.
\end{abstract}

\category{CR-number}{subcategory}{third-level}

% general terms are not compulsory anymore,
% you may leave them out
\terms
term1, term2

\keywords
keyword1, keyword2

\section{Introduction}\label{sec-introduction}
In recent years, Android has emerged as one of the most popular and
important platform for smart device development. Although most
Android applications are written in the Java programming language, they
are not compiled to traditional Java bytecode, instead, they
are compiled to a new Dalvik instruction set architecture
introduced by Google (the .dex or .odex format), which can
be executed by the brand-new virtual machine called Dalvik. The most notable
difference between Java bytecode and Dalvik bytecode is that the 
former is a stack-based architecture whereas the latterh is
a register-based one. 

Although empirical studies \cite{} have revealed that the register-based
instruction-set can be implemented more efficiently by the underlying
virtual machine, Android's new instruction set architecture 
does pose new challenge to program analysis and transformation: it's hard
to apply traditional Java bytecode analysis tools to analyze
and transform the new Dalvik code. And it's also nontrivial and
error-prone to port traditional program analysis tool for
Java bytecode
to Dalvik, due to the different architecture design philosophy. This
fact has hindered the implementation of many tools on Dalvik system, such
as bug-finding tools, symblic executors, optimizations, and profilers, and so on.

To address these challenges and difficulties, there have been two
directions of research efforts recently:
\begin{itemize}
  \item translating Dalvik bytecode to some other ISAs and
    performing analysis on those target ISAs; and
  \item implementing native analysis on Dalvik bytecode directly.
\end{itemize}

With the first research effort, the Dalvik programs are translated
to some existing platforms, on which all the program
analysis and transformation can be performed, in an indirect manner.
For instance, the Dare project \cite{dare2012} translates Dalvik
bytecode to standard Java bytecode, so that all program
analysis and transformation can be performed on the
generated bytecode. Other example projects include
compilers to LLVM \cite{}; or Dexpler, a translator from Dalvik
to Jimple and 
Soot \cite{}. Although all these researches and tools are successful
in some facet, there are severe drawbacks with these researches: first, all
Second, to our best knowledge, none of exiting research projects
has achieved completeness of the translation; for instance, according
to the experimental result from the Dare \cite{dare2012}, although
Dare successfully convert 99\% of 1100 applications
, but it's still unclear whether or not it can support the full Dalvik.

With the second research efforts, native tools are built to
analyze and transform bare Dalvik bytecode directly. Examples of
such cases include Google's
Dexdump which provides disassembly facilities for dex files. Other
similar research and tools 
include baksmali \cite{}, apktools \cite{}, jeb \cite{}, etc.. However, most of
existing researches and tools are disassembly or reverse engineering-oriented, 
instead of for program analysis. And it's hard, if not
impossible, to use them as general-purpose
program analysis and transformation framework without extensive
modification or rewriting.

So, the primary goal of the work described in this paper is to design
and implement a new open-source general-purpose framework named
Carbon for
Android application analysis and transformation. The design philosophy
of Carbon is to follow closely with current well-established
compiler and virtual
machine practice, so that developers can investigate and experiment
their algorithms on Carbon easily. Furthermore, a key requirement
for Carbon is to
represent the full Dalvik ISA, no subset, superset or other
variants will be supported, so that Carbon will be a coherent
framework. The current development of Carbon is rather 
stable and. This paper introduces the design and implementation
details and tradeoffs of Carbon, and lessons we learned in implementing it.

To summarize, the key technical contribution and highlight
of this paper is the design and implementation of a new
framework for Android application analysis and transformation. We
believe that we are the first to propose such a general-purpose
Android program analysis framework. Our experiments results
suggests that Carbon can be used in other circumstances.

The rest of this paper is organized as follows. Section \ref{sec-design}
introduces the overall design of Carbon; section \ref{}; and section
\ref{sec-conclusion} concludes.

\section{Design Overview}\label{sec-design}
We start, in this section, with an overview of the overall
design philosophy and high-level structure of Carbon. And
in the next several sections, we'll introduce more design
and implementation details.

The first decision we must make is to determine the format
of target applications Carbon can process. We have decided
to use the .apk file format. There are two reasons for this
decision: first, although most Android applications
are developed in Java, they are compiled to .dex format
before installation, so most application that can be found
on the Android macket an 

The second decision we must make is the language chosen
to write Carbon itself. Although we can, in principal, use 
any programming languages to finish this task, we
have chosen Java as the implementation language. The main
reason here is that we'd like to bootstrap Carbon (that
is, run Carbon with Carbon) in one of our on-going project
in which we are extending
Carbon with full-featured virtual
machine capability support. Of course, choosing Java also makes
it easy and straightforward for other developers to
write their own program analysis and transformation
on Carbon.

As a high-level intermediate representation, abstract
syntax trees are used to represent Dalvik abstract
syntax. The abstract syntax tree encodes honestly
the internal structures of an APK executable: class, 
methods, intructions, etc.. The details of this
representation will be described in the next section.

Except for the abstract syntax tree, we have decided
to use a powerful yet simple control-flow graph
data structure as the primary intermediate representation for
Carbon. On one hand, the
control-flow graph make it easy to calculate various
control-flow and data-flow information, meanwhile, it
hiddens many annoying details by de-sugaring many
language features
during the translation from the abtract syntax tree
to the control-flow graph. Section \ref{sec-cfg} will
describe the design details of the control-flow graph.

As we mentioned in section \ref{sec-introduction}, the
most important design philosophy of Carbon is to make
use of the well-established design practices of
compilers or virtual machines, so that developers
of various program analysis and transformation algorithm
can start their work as conveniently as possible, so we
have decided to 

\section{Front-end and Abstract Syntax Trees}
In this section, we introduce the design and implementation
details of the front-end part of Carbon: the lexer,
the parser, and the abstract syntax trees.

\subsection{Front-end}
Carbon takes an APK file as its input, so the first phase
will perform lexical and syntactic analysis on the given
input file. As the APK file format is in binary form, and 
Dalvik contains more than 200 different
instructions, so, writing a lexer and a parser from
scratch for any APK file
is a nontrivial and error-prone work. So we have decided
to make use some mature automatic tools to finish
this task. The tool that is used in Carbon is the antlr3
lexer and parser generators. 

\subsection{Bytecode Verifiers}
Just as a traditional Java virtual machine, the Dalvik virtual
machine will perform bytecode verification, when a class is first
parsed and loaded into a running virtual machine. Although
the underlying architecture are different, the Dalvik virtual
machine does a similar work as a JVM, with similar reasons: optimizations,
precise GC, intra-application security and failure analysis. However, according
to the specification from Google, all these verifications can
be enabled or disabled for some or all classes. So, to
make things simpler, we haven't put it a priority
to write 
a bytecode verifier from scratch, Carbon users who want to
verify their files can use the offical
Google's verifier. But it should be of no special technical
difficulty for us to write our own bytecode verifier.

\subsection{Abstract Syntax Trees}
The design of the abstract syntax trees in Carbon follows
the syntax of the .dex file format closely. Although the
design and implementation of Carbon's abstract syntax trees
repect the well-established compiler practice, however, there is
some key difference between Carbon's abstract syntax trees and
conventional ones found in exiting Java compilers, due to the file
structure difference between Java's .class file and Android's
.dex file. Traditionally, a Java compiler will store just
one class in any .class file. In contrast, in order to save
space, Android will store all classes in just one .dex file
by merging all same structures from all class files. To reflect
this, we have decided to use a forest data structure to
encode the whole program, with
each tree node in the forest encoding just one .class 
file. 

In each tree node in the graph, we have all the information
of a class: its full qualified name, access properties, 
instance fields, all methods, etc.. And the rule can
be given as:


For each method, there are its name, prototypes, all
statements, etc.. However, some information are not
compiled from the Java source but are generated by
the Java or by the Google dx compiler, one such example is
the attribute of numbers of virtual registers in given method. The
data structure definition are given in the figure ...

Statements are the most complex part in the abstract
syntax tree, because it must take into account more than
200 instructions defined by the Dalvik ISA. Typical
statemens can be given in the figure ...

To summarize, due to the complexity of the Dalvik instruction set
and the corner cases in the syntax, the abstract syntax
tree has been one of the most complex part. In our
experience, the current implementation consists of 
more than 5000 lines of Java code and continue to
increase rapidly.


\section{Factored Control-flow Graph}\label{sec-cfg}
This section introduces the control-flow graph, the key
intermediate representation used in Carbon. We have two
purpose in designing the 

\section{Experiments and Evaluations}\label{sec-exp}
In this section, we report the current implementation
status of Carbon, the experiments performs, the evalutions
and conclusions can be drawn, and the lessions learned. We
have several goals in reports all these facts: first, we
want to shed some light on the 

\section{Conclusion}\label{sec-conclusion}

Summary...

\acks

This project is supported by a CCF-Tencent
Open Research Fund, number CCF-Tencent AGR20130105. Any
opinions, findings, and recommendations in this paper are those
of the authors, and do not
necessarily reflect the views of the China Computer Federation, or
Tencent.

% We recommend abbrvnat bibliography style.

\bibliographystyle{abbrvnat}

% The bibliography should be embedded for final submission.

\begin{thebibliography}{}
\softraggedright

\bibitem[Smith et~al.(2009)Smith, Jones]{smith02}
P. Q. Smith, and X. Y. Jones. ...reference text...

\bibitem[]{dare2012}
Damien Octeau, Somesh Jha and Patrick McDaniel. Retargeting 
Android Applications to Java Bytecode. 20th International 
Symposium on the Foundations of Software Engineering (FSE). Cary, NC. November 2012.

\end{thebibliography}


\end{document}

%                       Revision History
%                       -------- -------
%  Date         Person  Ver.    Change
%  ----         ------  ----    ------

%  2013.06.29   TU      0.1--4  comments on permission/copyright notices

