%-----------------------------------------------------------------------------
%
%               Template for sigplanconf LaTeX Class
%
% Name:         sigplanconf-template.tex
%
% Purpose:      A template for sigplanconf.cls, which is a LaTeX 2e class
%               file for SIGPLAN conference proceedings.
%
% Guide:        Refer to "Author's Guide to the ACM SIGPLAN Class,"
%               sigplanconf-guide.pdf
%
% Author:       Paul C. Anagnostopoulos
%               Windfall Software
%               978 371-2316
%               paul@windfall.com
%
% Created:      15 February 2005
%
%-----------------------------------------------------------------------------


\documentclass[preprint, cm]{sigplanconf}

% The following \documentclass options may be useful:

% preprint      Remove this option only once the paper is in final form.
% 10pt          To set in 10-point type instead of 9-point.
% 11pt          To set in 11-point type instead of 9-point.
% authoryear    To obtain author/year citation style instead of numeric.

\usepackage{amsmath}


\begin{document}

\special{papersize=8.5in,11in}
\setlength{\pdfpageheight}{\paperheight}
\setlength{\pdfpagewidth}{\paperwidth}

\conferenceinfo{CONF 'yy}{Month d--d, 20yy, City, ST, Country}
\copyrightyear{20yy}
\copyrightdata{978-1-nnnn-nnnn-n/yy/mm}
\doi{nnnnnnn.nnnnnnn}

% Uncomment one of the following two, if you are not going for the
% traditional copyright transfer agreement.

%\exclusivelicense                % ACM gets exclusive license to publish,
                                  % you retain copyright

%\permissiontopublish             % ACM gets nonexclusive license to publish
                                  % (paid open-access papers,
                                  % short abstracts)

%\titlebanner{banner above paper title}        % These are ignored unless
%\preprintfooter{This project is supported by a CCF-Tencent
%Open Research Fund, number CCF-Tencent AGR20130105.}   % 'preprint' option specified.

\title{Carbon: an Open Framework for Analysis and Transformation of Android Applications}
%\subtitle{Subtitle Text, if any}

\authorinfo{Name1 \and Name2 \and Name3}
           {School of Software Engineering\\
           University of Science and Technology of China}
           {Email1}
%\authorinfo{Name2\and Name3}
%           {Affiliation2/3}
%           {Email2/3}

\maketitle

\begin{abstract}
In recent years, Android platform has emerged as one of the most
important open mobile platforms. However, as Android has introduced
new file format and new Dalvik instruction set, existing binary-level
program analysis and transformation tools can not be used directly on
Android application binaries.

This paper presents the design and implementation of Carbon: a
new open framework for analysis and transformation of Android
applications at binary level. First, Carbon is an intermediate
representation suitable for program analysis and transformation, in
its current status, Carbon contain both high-level IR such as
abstract syntax trees, and low-level IR such as control-flow graphs. Second, Carbon
contains a front-end which can parse any given binary APK files and produce
abstract syntax trees and control-flow graphs. Third, Carbon is
now being used in several of
our ongoing projects, such as a symbolic executor for Dalvik; our
preliminary experiments and experience shows that Carbon is
an effective Android application binary analysis and transformation
framework, and we believe it can usefully be applied in other
situations as well.
\end{abstract}

\category{CR-number}{subcategory}{third-level}

% general terms are not compulsory anymore,
% you may leave them out
\terms
term1, term2

\keywords
keyword1, keyword2

\section{Introduction}
In recent years, Android has emerged as one of the most popular and
important platform for smart device development. Although most
Android applications are written in the Java programming language, they
are not compiled to traditional Java bytecode, instead, they
are compiled to a new Dalvik instruction set architecture
introduced by Google (the .dex or .odex format), which can
be executed by the brand-new virtual machine called Dalvik. The most notable
difference between Java bytecode and Dalvik bytecode is that the 
former is a stack-based architecture whereas the latterh is
a register-based one. 

Although empirical studies \cite{} have revealed that the register-based
instruction-set can be implemented more efficiently by the underlying
virtual machine, Android's new instruction set architecture 
does pose new challenge to program analysis and transformation: it's hard
to apply traditional Java bytecode analysis tools to analyze
and transform the new Dalvik code. And it's also nontrivial and
error-prone to port traditional program analysis tool for
Java bytecode
to Dalvik, due to the different architecture design philosophy. This
fact has hindered the implementation of many tools on Dalvik system, such
as bug-finding tools, symblic executors, optimizations, and profilers, and so on.

To address these challenges and difficulties, there have been two
directions of research efforts recently:
\begin{itemize}
  \item translating Dalvik bytecode to some other ISAs and
    performing analysis on those target ISAs; and
  \item implementing native analysis on Dalvik bytecode directly.
\end{itemize}

With the first research effort, the Dalvik programs are translated
to some existing platforms, on which all the program
analysis and transformation can be performed, in an indirect manner.
For instance, the Dare project \cite{dare2012} translates Dalvik
bytecode to standard Java bytecode, so that all program
analysis and transformation can be performed on the
generated bytecode. Other example projects include
compilers to LLVM \cite{}; or Dexpler, a translator from Dalvik
to Jimple and 
Soot \cite{}. Although all these researches and tools are successful
in some facet, there are severe drawbacks with these researches: first, all
Second, to our best knowledge, none of exiting research projects
has achieved completeness of the translation; for instance, according
to the experimental result from the Dare \cite{dare2012}, although
Dare successfully convert 99\% of 1100 applications
, but it's still unclear whether or not it can support the full Dalvik.

With the second research efforts, native tools are built to
analyze and transform bare Dalvik bytecode directly. Examples of
such cases include Google's
Dexdump which provides disassembly facilities for dex files. Other
similar research and tools 
include baksmali \cite{}, apktools \cite{}, jeb \cite{}, etc.. However, most of
existing researches and tools are disassembly or reverse engineering-oriented, 
instead of for program analysis. And it's hard, if not
impossible, to use them as general-purpose
program analysis and transformation framework without extensive
modification or rewriting.

So, the primary goal of the work described in this paper is to design
and implement a new open-source general-purpose framework named
Carbon for
Android application analysis and transformation. By doing so, we
hope to gain deeper understanding of program analysis and
transformation techniques
for Android applications. The design philosophy
of Carbon is to stay close to current mature compiler and virtual
machine practice, so that developers can investigate and experiment
their algorithms. Furthermore, a key requirement for Carbon is to
represent the full Dalvik ISA, no subset, superset or other
variants will be supported, so that Carbon will be a coherent
framework. 




The rest of this paper is organized as follows. Section \ref{sec-design}
introduces the overall design of Carbon; section \ref{}; and section
\

\section{Design Overview}\label{sec-design}

\subsection{Front-end}

\section{Summary}


\section{Summary}\label{sec-summary}

Summary...

\acks

This project is supported by a CCF-Tencent
Open Research Fund, number CCF-Tencent AGR20130105. Any
opinions, findings, and recommendations in this paper are those
of the authors, and do not
necessarily reflect the views of the China Computer Federation, or
Tencent.

% We recommend abbrvnat bibliography style.

\bibliographystyle{abbrvnat}

% The bibliography should be embedded for final submission.

\begin{thebibliography}{}
\softraggedright

\bibitem[Smith et~al.(2009)Smith, Jones]{smith02}
P. Q. Smith, and X. Y. Jones. ...reference text...

\bibitem[]{dare2012}
Damien Octeau, Somesh Jha and Patrick McDaniel. Retargeting 
Android Applications to Java Bytecode. 20th International 
Symposium on the Foundations of Software Engineering (FSE). Cary, NC. November 2012.

\end{thebibliography}


\end{document}

%                       Revision History
%                       -------- -------
%  Date         Person  Ver.    Change
%  ----         ------  ----    ------

%  2013.06.29   TU      0.1--4  comments on permission/copyright notices

